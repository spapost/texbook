
% % ==== Подпараграф =======================================================
	\section{Форматы вывода}\label{Sec:OutFormats}

Система \TeX\ может переводить текст в формат DVI (Device Independent), PDF (Portable Document Format) и PS (PostScript). Все эти форматы являются плат\-фор\-мно-независимыми, однако каждый из них имеет свои особенности, которые зачастую предопределяют их использование.

\textbf{Формат DVI}\index{Формат вывода!DVI} очень хорошо подходит для предварительного просмотра текста в процессе набора, вёрстки и пробной печати. Однако, хотя он и называется <<device independent>> и одинаково хорошо приспособлен для вывода на печать или экран, при переносе с одной системы на другую может некорректно отображаться, поскольку зависит от настроек программной среды: наличия шрифтов, вспомогательных средств и~пр.

\textbf{Язык PostScript}\index{Формат вывода!PostScript}, базирующийся на языке Forth, является одним из плат\-фор\-мно-не\-за\-ви\-си\-мых языков описания страниц и использует векторную форму описания. При этом подразумевается, что устройство (будь то дисплей, принтер, плоттер) самостоятельно выполняет преобразование векторного представления в растровое с учётом своей разрешающей способности.

Для каждого из устройств результирующее растровое изображение будет различным, однако размеры, положение, координаты вершин объектов и прочие геометрические параметры обязаны совпадать. Компания Adobe обладает лицензией на все реализации  PostScript, что, с одной стороны, значительно влияет на цену устройств, поддерживающих этот язык, а с другой "--- обеспечивает почти полную совместимость с любой компьютерной платформой. Чрезвычайно широкое распространение PostScript нашел на платформе Apple Macintosh.

Язык имеет средства описания прямых линий, углов, прямоугольников и кривых третьего порядка, возможность отображения линий любой толщины и любого цвета (цвет может быть представлен в моделях Pseudo-Greyscale, RGB и CMYK). Текст полностью интегрирован с графикой, а текстовые символы, как предопределённые, так и пользовательские, могут быть объектами любого графического оператора.

PostScript позволяет описывать растровые изображения любого разрешения с использованием различных цветовых моделей. Общая система координат поддерживает комбинации следующих преобразований: масштабирование, поворот, зеркальное отображение. Преобразования применяются одновременно ко всем элементам страницы, включая текст, графические примитивы и растровые изображения.

Вторая версия "--- PostScript Level 2 (старое название Sup\-ra) "--- обеспечила, кроме того, платформно-независимое отображение цвета, сжатие данных и улучшенные алгоритмы преобразования полутоновых изображений.

В последней реализации "--- Adobe PostScript~3 "--- ядро интерпретатора претерпело значительные изменения, по сравнению с предыдущей версией увеличилась производительность.

Encapsulated PostScript был создан для обеспечения межплатформенной переносимости рисунков в формате PostScript. Основное отличие EPS заключается в том, что возможности формата ограничиваются описанием одной страницы. Объекты страницы EPS не влияют на изображения, находящиеся за пределами этой страницы. Файл EPS может содержать растровый эскиз изображения для предварительного просмотра (чаще всего в формате TIFF).

Для просмотра файлов PS (EPS) необходима программа-интерпретатор языка PostScript, например GhostScript и программа просмотра GSView, или редакторы, воспринимающие данный формат "--- Adobe Illustrator, Corel Draw, Adobe Acrobat. В числе прочих данный формат воспринимает и Adobe Acrobat Reader.

\index{Формат вывода!PDF}\textbf{Формат PDF} был создан компанией Adobe как универсальный формат для хранения и переноса данных на разных вычислительных платформах. Файлы формата PDF <<весят>> больше, чем файлы DVI, но основным их преимуществом является <<истинная>> платформенная независимость (проблема с наличием шрифтов решается путём интегрирования использованных символов шрифта в документ, что положительно сказывается на внешнем виде и отрицательно "--- на объёме) и неизменность при печати. Являясь по сути графическим объектом, этот формат прекрасно отображает графику (рисунки многих форматов, от JPEG до EPS), поддерживает <<художественное>> оформление текста и создание презентаций с гиперссылками.

В отличие от формата DVI, в файл формата PS, EPS или PDF можно внедрить символы используемых шрифтов (или, в случае PS/EPS "--- сохранить символы как кривые), что позволяет на \textit{любом} компьютере или печатающем устройстве увидеть в точности такой результат, какой необходим "--- даже если соответствующая гарнитура отсутствует.



\endinput
