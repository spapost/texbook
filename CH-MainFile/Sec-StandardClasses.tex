
% % ==== Параграф ==========================================================
	\section{Стандартные классы}\label{Sec:StandardClasses}

В предыдущих примерах мы использовали один из стандартных классов, которые предопределены в системе \LaTeX. Эти классы были изначально созданы на основе американских стандартов и общепринятых правил и, поэтому, часто не самым лучшим образом подходят под отечественные реалии. Тем не менее, использование стандартного класса для абсолютного большинства пользователей является необходимым, поскольку написание своего класса процесс очень сложный и абсолютно бессмысленный, если, например, нужно сделать какую-то учебную работу. 

Поэтому мы рассмотрим основные стандартные классы, которые описаны в системе \LaTeX, а в последующих главах будем учиться их настраивать. 

Класс \texttt{article}\index{article}\index{Документ!класс!article} "--- статья "--- лучше всего подходит для набора небольших статей, курсовых работ и проектов, заметок, докладов и~т.~п. В этом классе по умолчанию установлен односторонний набор (поэтому поля на всех страницах одинаковы), сквозная нумерация всех нумеруемых объектов (сносок, формул, таблиц, рисунков). Наверно, в большинстве случаев это наиболее приемлемый класс, поэтому именно он рекомендуется в студенческих работах.

Класс \texttt{report}\index{report}\index{Документ!класс!report} "--- отчёт "--- предназначен для набора отчётов по научной и исследовательской работе. Этот класс поддерживает по умолчанию двусторонний набор, но ещё не дотягивает до того, чтобы набирать книги без сложных предварительных настроек.

Класс \texttt{book}\index{book}\index{Документ!класс!book} "--- книга "--- в большинстве случаев используется для создания брошюр и книг. При использовании этого класса документ будет оформлен как книга: двусторонняя разметка страницы (учитываются правые и левые поля, нумерация страниц), колонтитулы, особенности оформления рубрик (например, главы автоматически начинаются с новой нечётной страницы) и многое другое.

Каждый класс имеет ряд определённых особенностей, которые были заложены при их создании, например, глубину вложения и оформление разделов, их нумерацию, оформление титульных листов, оглавлений и многое другое. Однако эти классы построены по одной схеме, да и отличие состоит не в каких-то ключевых моментах, а скорее в некоторых частностях. Тем не менее создание классов представляет собой сложный и очень трудоёмкий процесс, впрочем, как и их видоизменение. Приведённые выше основные классы документа имеют свои возможные \verb|<параметры класса>|, которые более подробно рассмотрены в приложениях в конце книги (стр.~\pageref{struct::ClassOptions->Appendix}). Некоторые из этих свойств будут вводиться по ходу изложения.


\endinput
